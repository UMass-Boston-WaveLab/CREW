% !TEX TS-program = pdflatex
% !TEX encoding = UTF-8 Unicode

% This is a simple template for a LaTeX document using the "article" class.
% See "book", "report", "letter" for other types of document.

\documentclass{allertonproc} 

%%% Examples of Article customizations
% These packages are optional, depending whether you want the features they provide.
% See the LaTeX Companion or other references for full information.
\usepackage{cite}
\usepackage{graphicx} % support the \includegraphics command and options
\usepackage{amsmath}
% \usepackage[parfill]{parskip} % Activate to begin paragraphs with an empty line rather than an indent

%%% PACKAGES
\usepackage{booktabs} % for much better looking tables
\usepackage{array} % for better arrays (eg matrices) in maths
\usepackage{paralist} % very flexible & customisable lists (eg. enumerate/itemize, etc.)
\usepackage{verbatim} % adds environment for commenting out blocks of text & for better verbatim
\usepackage{subfig} % make it possible to include more than one captioned figure/table in a single float
% These packages are all incorporated in the memoir class to one degree or another...

%%% The "real" document content comes below...

\title{SPATIAL CHANNEL PREDICTION FOR EAVESDROPPING ON WIRELESS FADING BASED KEY GENERATION} %this title is not bad
\author{Eric Brown, Clara Gamboa, and K.~C.~Kerby-Patel\\University of Massachusetts Boston, Boston, MA\\kc.kerby-patel@umb.edu} %the convention is that the advisor's name goes last
%also have to put in our UMB affiliation blurb
\date{} % Activate to display a given date or no date (if empty),
         % otherwise the current date is printed 

\begin{document}
\maketitle
\bibliographystyle{ieeetran}

\begin{abstract}

Physical layer key generation techniques based on wireless channel fading are generally considered to be secure as long as any eavesdroppers are separated from the terminals by a distance greater than the channel correlation length.  In fact, this is true only if the channel observations are jointly Gaussian random variables.  A channel's transfer function is determined by the physical scatterers in that particular channel realization, and information about the communicating parties' channel may persist over significantly longer distances than the correlation length if the physical parameters of the channel are slowly varying.  Linear prediction has previously been employed to predict samples of the channel transfer function ahead in time.  In this work, we demonstrate that it is also possible to predict the channel transfer function in space, over distances larger than the correlation length.  This leads us to conclude that a more pessimistic ``safe'' eavesdropper distance for wireless fading based key generation is needed.
\end{abstract}
%we will have to compute the correlation length from the observed channel covariance matrix.  

%\textbf{Key words:} -- wireless, channel, fading, key generation, physical layer, security

\section{Introduction}
It has been demonstrated \cite{azimisadjadi2007, bloch2008, mathur2008, ye2010} that two parties' reciprocal observations of a fading wireless channel can be used as a common source of randomness to generate symmetric encryption keys in cases where two parties cannot pre-arrange or securely exchange keys.  Each party transmits a predetermined waveform, which the other party receives and uses to determine the channel transfer function.  These measurements are assumed to be the same in both directions if completed within a short period of time, because of the channel's reciprocity.  Since the channel transfer function is also position dependent, eavesdroppers at other locations observe a different channel transfer function and the legitimate nodes' channel measurements are usually assumed to be secret.  In order for this type of link signature based keying to be useful as a practical encryption method, the secrecy of the channel measurements must be quantified.  This work examines how (and under what circumstances) an eavesdropper could predict the channel observations of the communicating parties.

The security of this key generation method is based on the assumption that if an eavesdropper is located farther than a correlation length from the communicating parties, it will have essentially no information about the symmetric key generated from their reciprocal channel measurements \cite{azimisadjadi2007, bloch2008, mathur2008, ye2010, he2013}.  Since the channel correlation function in many channels falls off quickly with distance \cite{jakes1974}, the minimum secure distance is often stated to be a half wavelength.  However, mutual information between two channel observation locations may be high even if the channel correlation function is low or has a minimum.  The channel transfer function is a deterministic function of physical environmental parameters.  Although these environmental parameters may be unknown, in they often vary quite slowly with distance compared to the correlation function of the channel \cite{jakes1974, duel-hallen2007}.  The mutual information between the channel parameters at two locations represents an upper bound on the mutual information between channel observations at those locations \cite{kckpVTC2015}.  Such information may persist over distances longer than the correlation length, so security claims based on the correlation length are inappropriate.  In this work, we apply channel prediction techniques \cite{duel-hallen2007} to investigate the distance over which it is possible to predict the channel transfer function.

A similar but narrower problem was identified by He \emph{et al}.~in \cite{he2013}.  That work pointed out that the correlation functions of some channels have broad peaks in the spatial domain, and fading-based security using such channels is vulnerable to correlation-based attacks based on linear minimum mean-square error estimation.  By contrast, the minimum secure distance discussion we present is not based on the width of the correlation function but on the mutual information between separated channel observations, which can persist even when the correlation function is highly oscillatory and has a narrow peak.  As a result, it applies to a more general set of channels and is not limited to those with low angular spread.  In order to predict a channel with wide angular spread, the eavesdropper in this work is assumed to use a spectral estimation technique.

Section \ref{mutualinfo} briefly summarizes the argument for a mutual information based minimum secure distance that was presented in \cite{kckpVTC2015}.  Next, in Section \ref{prediction}, we apply long-range prediction techniques to a series of spatial samples in order to predict the channel transfer function over distances greater than the correlation length.  In Section \ref{simresults}, we apply this technique to a simple simulated channel to examine the dependence of prediction length on the properties of the channel and the eavesdropper array. 

\section{Mutual Information vs.~Correlation}\label{mutualinfo} %this needs a better section heading?
The ability of an eavesdropper to estimate the communicating parties' encryption key depends on the mutual information between the eavesdropper's channel observations and the legitimate nodes' observations.  In the literature, the channel correlation function is usually used as a proxy for the mutual information.  In \cite{kckpVTC2015} it was argued that this leads to an inappropriately optimistic minimum secure eavesdropper distance, and an eavesdropper may be able to predict the legitimate nodes' channel observations despite being more than a correlation length away.  This section briefly summarizes that argument, which is the motivation for the channel prediction simulation that follows.

If two random variables are jointly Gaussian, their mutual information is a monotonic function of their correlation function.  Thus, observations of a wireless channel at two locations, $h(\mathbf{r})$ and $h(\mathbf{r}+\mathbf{\Delta r})$, have mutual information given by 
\begin{equation}\label{mutualinformationgauss}
I(h(\mathbf{r}),h(\mathbf{r}+\mathbf{\Delta r})) = -\frac{1}{2}\ln\left(1-\frac{R_h(\mathbf{r},\mathbf{r}+\mathbf{\Delta r})}{\sigma_h^2}\right)
\end{equation}
where $R_h$ is the correlation function, defined as usual, and $\sigma_h^2$ is the average power.
\begin{equation}
R_h(\mathbf{r},\mathbf{r}+\mathbf{\Delta r}) = E[h(\mathbf{r})h^*(\mathbf{r}+\mathbf{\Delta r})]
\end{equation}

If the two observations are not jointly Gaussian, the expression given in (\ref{mutualinformationgauss}) is merely a lower bound on the mutual information.  In general, the channel transfer function at a particular location is a deterministic function of a set of environmental parameters $\boldsymbol{\theta}$ at that location. The environmental parameters may be considered to be a random variable. The data processing inequality can be applied to find that the mutual information of channel transfer function values at two locations with environmental parameters $\boldsymbol{\theta_1}$ and $\boldsymbol{\theta_2}$ is bounded by the mutual information of the environmental parameters.
\begin{equation}
I(h(\boldsymbol{\theta_1},\mathbf{r_1}); h(\boldsymbol{\theta_2},\mathbf{r_2}))\leq I(\boldsymbol{\theta_1}; \boldsymbol{\theta_2})
\end{equation}

Observations of the channel are given by $\tilde{h}(\boldsymbol{\theta})=h(\boldsymbol{\theta})+n$, where $n$ is additive white Gaussian noise.  The mutual information between two observations is limited by the mutual information between the channel transfer function itself at those two locations, again by the data processing inequality.
\begin{equation}
I(\tilde{h}(\boldsymbol{\theta_1},\mathbf{r_1}); \tilde{h}(\boldsymbol{\theta_2},\mathbf{r_2}))\leq I(h(\boldsymbol{\theta_1},\mathbf{r_1}); h(\boldsymbol{\theta_2},\mathbf{r_2}))
\end{equation}

If the observations are close enough to each other spatially that the environmental parameter vector $\boldsymbol{\theta}$ does not change appreciably, the upper bound on the mutual information between two observations is the entropy of $\boldsymbol{\theta}$.
\begin{equation}
I(\tilde{h}(\boldsymbol{\theta},\mathbf{r_1}); \tilde{h}(\boldsymbol{\theta},\mathbf{r_2}))\leq H(\boldsymbol{\theta})
\end{equation}

This discussion has provided a lower bound and an upper bound on the mutual information between two observations of a wireless channel.  Clearly, significant room remains for variation between the two bounds.  However, it has been observed that environmental parameters are slowly varying spatially compared to the channel transfer function or its correlation function \cite{jakes1974, duel-hallen2007}.  This indicates that information about the legitimate nodes' channel may be available to the eavesdropper even if it is separated from them by more than a correlation length.   The correlation length merely indicates that the value of the correlation function is zero at a particular location.  

\section{Spatial Long-Range Channel Prediction Method}\label{prediction}
%belongs to Eric
As discussed in Section \ref{mutualinfo}, channel information may persist over distances greater than the correlation length. Thus, long-range prediction techniques \cite{a long range prediction paper} may be applied to predict the channel from spatial samples. Rather than using long-range prediction to predict the channel transfer function ahead temporally, as in \cite{duel-hallen2007}, we predict it spatially.  The structure of the problem is essentially the same, except for the dimensionality... (maybe also draw analogy to direction finding and describe how we're going to estimate theta and then reconstruct the channel)

Previous literature(CITE) has shown that this problem is conducive to a spectral estimation approach. This is due tot he fact that we are only attempting to predict a small number of wavelengths forward in space. (DUEL-HALLEN PAPER) States that channel can be very accurately estimated within a single doppler wavelength, which encompasses several carrier wavelengths. In temporal prediction applications, it has been observed that time variation of the amplitudes and frequencies in the sum-of-sinusoids model is on the order of 0.1 seconds \cite{duel-hallen2007} \emph{This may depend on how fast they assume the receiver is moving?} (WHY Do we need this?).  In order to assemble an acceptable set of data to solve a spectral estimation problem, we simulate array of spatial samples with regular spacing to collect information on the wireless channel.

We simulate the listener as a series of recievers along a line that points in the direction of the point we'd like to simulate. Since we assume that all sampling occurs simultaneously our equation is simplified. The Equation \ref{chan} describes a transfer function along a line.


\begin{equation}\label{chan}
\hat{H}= \sum_{n=1}^N \alpha_n e^{-j(k+k_d\cos\theta_n)d}
\end{equation}

 Where $\hat{H}$ is the channel estimate, $\alpha_n$ is the $n^{th}$ complex amplitude, N is the number of samples, $d$ is the distance between the samples, k is the wave number, and $k_d\cos\theta_n$ is the doppler wave number. 

Our unknowns in this system are the amplitude and the wave numbers. We make the assumption that the angle of arrival for all the listening samples is the same because they are all in the far field. This array of samples is of length $Nd+qd$, where $q$ is the number of sample lengths ahead we'd like to predict. With this channel prediction model we obtain a set of data points for the channel in each location, then use spectral estimation to project this sample out to the location of B.

The solution to the estimation problem can be broken into two parts. Estimating the amplitudes, then estimation the wave number $k$ from the resultant amplitude array. We used the rootMUSIC algorithm supplied in MATLAB to determine the amplitudes. RootMUSIC uses eigenvalue analysis of the channel's correlation matrix to estimate the signals frequency content. The specific command used estimates the discrete frequency spectrum, and provides the signal power.  In order to determine wavenumber we used the method described in (CITE paper cant find it)(Describe method we used, took a covariance matrix did stuff, least mean squares blah blah) 

  %say what we do and why (spectral estimation is ok because  parameters don't change very fast)


%describe the problem setup: channel, eavesdropper


%summary of math - MUSIC (cite MATLAB) and how to estimate the coefficients (cite paper where we got that technique)


\section{Spatial Long-Range Channel Prediction Simulated Results}\label{simresults}
%%%REFER TO VTC PAPER: "The CRLB expressions presented in \cite{**vtc**} were used to identify suitable parameters for a default simulation scenario.   ...more details and maybe graphs if we feel like it.  Putting this in its own subsection might be a good idea.



%belongs to Eric
%where is a good place for this sentence.
The channel model is shown in Figure (insert). Point A is the transmitter, several scatterers S, and a desired estimation location B. E is an array of sample readings taken in line with the listener. The model uses 10 scatterers, as it has been shown in (DUAL HALLEN citation) most real world channels contain less than 10 large scatterers. In our simulation we assume the sender A and all the scatterers reside in the far field. 
%in this section we apply the method from the previous section to a simulated channel

%describe the channel simulation (very similar to problem setup from previous section) - what did we assume about the problem geometry - how are the angles distributed? (uniformly) what is f_doppler/f_c?

%find average prediction length and how it varies for different values of $d$ (element spacing in E), $N$ (number of elements in E), and $S$ (number of scatterers) - include figures

These prediction lengths are unrealistic because the simulated channel doesn't change vs. position at all.  A realistic channel will change as the eavesdropper's position changes. 

\section{Conclusion}
%belongs to Clara
%we applied long-range prediction to estimate the channel transfer function at a location in space that is separated from the eavesdropper by more than a correlation length.  This indicates that the minimum safe distance for secure key generation based on wireless fading must be reexamined.

%our analysis is potentially pessimistic because we assume the channel parameters don't change within the sampling interval - however, it may be possible to extend the estimation to include both the channel parameters and their first derivatives.  CRLB will increase because it usually increases with additional parameters.

%future work: do a scenario where the eavesdropper moves through space and see if we can do spatiotemporal channel prediction simultaneously, see about 2D eavesdropper arrays and other problem geometries (what if all the scatterers are in an arc of the space instead of evenly distributed? does it matter?), do a security analysis by estimating the error of the prediction(?), demonstrate spatial channel prediction experimentally, investigate the effect of mutual coupling between array elements on the quality of the eavesdropper's predictions
\bibliography{PL-Library}{}
\end{document}