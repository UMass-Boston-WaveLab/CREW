% !TEX TS-program = pdflatex
% !TEX encoding = UTF-8 Unicode

% This is a simple template for a LaTeX document using the "article" class.
% See "book", "report", "letter" for other types of document.

\documentclass[11pt]{article} % This will be \documentclass{ieeetran} later

\usepackage[utf8]{inputenc} % set input encoding (not needed with XeLaTeX)

%%% Examples of Article customizations
% These packages are optional, depending whether you want the features they provide.
% See the LaTeX Companion or other references for full information.

%%% PAGE DIMENSIONS
\usepackage{geometry} % to change the page dimensions
\geometry{a4paper} % or letterpaper (US) or a5paper or....
% \geometry{margin=2in} % for example, change the margins to 2 inches all round
% \geometry{landscape} % set up the page for landscape
%   read geometry.pdf for detailed page layout information

\usepackage{graphicx} % support the \includegraphics command and options

% \usepackage[parfill]{parskip} % Activate to begin paragraphs with an empty line rather than an indent

%%% PACKAGES
\usepackage{booktabs} % for much better looking tables
\usepackage{array} % for better arrays (eg matrices) in maths
\usepackage{paralist} % very flexible & customisable lists (eg. enumerate/itemize, etc.)
\usepackage{verbatim} % adds environment for commenting out blocks of text & for better verbatim
\usepackage{subfig} % make it possible to include more than one captioned figure/table in a single float
% These packages are all incorporated in the memoir class to one degree or another...

%%% HEADERS & FOOTERS
\usepackage{fancyhdr} % This should be set AFTER setting up the page geometry
\pagestyle{fancy} % options: empty , plain , fancy
\renewcommand{\headrulewidth}{0pt} % customise the layout...
\lhead{}\chead{}\rhead{}
\lfoot{}\cfoot{\thepage}\rfoot{}

%%% SECTION TITLE APPEARANCE
\usepackage{sectsty}
\allsectionsfont{\sffamily\mdseries\upshape} % (See the fntguide.pdf for font help)
% (This matches ConTeXt defaults)

%%% ToC (table of contents) APPEARANCE
\usepackage[nottoc,notlof,notlot]{tocbibind} % Put the bibliography in the ToC
\usepackage[titles,subfigure]{tocloft} % Alter the style of the Table of Contents
\renewcommand{\cftsecfont}{\rmfamily\mdseries\upshape}
\renewcommand{\cftsecpagefont}{\rmfamily\mdseries\upshape} % No bold!

%%% END Article customizations

%%% The "real" document content comes below...

\title{On the Vulnerability of Wireless Fading Key Generation to Eavesdropping via Channel Prediction} %this title is not bad
\author{Clara Gamboa, Eric Brown, K.C. Kerby-Patel}  %the convention is that the advisor's name goes last
%also have to put in our UMB affiliation blurb
%\date{} % Activate to display a given date or no date (if empty),
         % otherwise the current date is printed 

\begin{document}
\maketitle

\section{Abstract}

Physical layer key generation techniques based on wireless channel fading are generally considered to be secure as long as any eavesdroppers are separated from the terminals by a distance greater than the channel correlation length.  This assertion depends on an assumption that the channel is ergodic, while many real-world channels are not ergodic.  Linear prediction has previously been employed to predict samples of the channel transfer function ahead in time.  In this work, we demonstrate that it is possible to predict the channel transfer function in space as well, over distances much larger than the correlation length.  %we will have to compute the correlation length from the observed channel covariance matrix.  
This leads us to conclude that a more pessimistic ``safe'' eavesdropper distance for wireless fading based key generation is needed.

\textbf{Key words:} -- wireless, channel, fading, key generation, physical layer, security

\section{Introduction}

It has been demonstrated \cite{anyPLkeygenpaper} that two parties' reciprocal observations of wireless channel fading can be used to generate symmetric encryption keys on the fly in a situation where two parties cannot pre-arrange keys. 
%possible applications
%references about the current state of this technique 
%security claims and their reasoning - implicit assumption of ergodicity by depending on the channel's correlation length
However, it has been shown that real-world wireless channels are typically not ergodic vs. time \cite{isukapalli}.  In this work, we establish that the spatial average of such a channel's correlation function is not equal to the ensemble average, meaning that then channel is also not ergodic vs. space.  This indicates that security claims based on the correlation length are inappropriate.  %to find out what level of security we can actually expect, we apply channel prediction techniques \cite{any duel-hallen paper} to investigate the distance over which an eavesdropper could predict the channel transfer function.

Section \ref{ergodicity} applies the method of \cite{isukapalli} to show that the spatial average of a channel's correlation function is not generally equal to the ensemble average, indicating that the channel is not spatially ergodic. Next, in Section \ref{prediction}, we examine the possibility of predicting the channel transfer function over distances greater than the correlation length by applying long-range prediction techniques.


\section{Spatial Ergodicity Analysis}\label{ergodicity}

Citing Isukapalli paper and explaining the math he used.

\section{Spatial Long-Range Channel Prediction Method}\label{prediction}

We are doing things differently. We take the channel model that is usually a fuction of time and align it along a listening array and transform it into a function of space. (This needs to be explained as the mathematical equations are given. Describe the variables and whatnot)

\section{Spatial Long-Range Channel Prediction Simulated Results}

In this section we will go on to describe how we set about creating a channel model and the methods we used. (LPC, ARYule, MATLAB code explanations and code may be used here.) This section should have graphs that show our channel model and explain the methodology clearly. We are using the sum of sinusoids method to assemble a wireless channel with "S" scatterers, "N" number of array listening points etc... In this section we also discuss how we went about our predictions and how far out we can reasonably predict. How many times were the tests run?

\section{Conclusion}
Is it broken? Is it safe? Can we be sure? What's the next step?


More text.

\end{document}
