
%% bare_conf.tex
%% V1.4
%% 2012/12/27
%% by Michael Shell
%% See:
%% http://www.michaelshell.org/
%% for current contact information.
%%
%% This is a skeleton file demonstrating the use of IEEEtran.cls
%% (requires IEEEtran.cls version 1.8 or later) with an IEEE conference paper.
%%
%% Support sites:
%% http://www.michaelshell.org/tex/ieeetran/
%% http://www.ctan.org/tex-archive/macros/latex/contrib/IEEEtran/
%% and
%% http://www.ieee.org/

%%*************************************************************************
%% Legal Notice:
%% This code is offered as-is without any warranty either expressed or
%% implied; without even the implied warranty of MERCHANTABILITY or
%% FITNESS FOR A PARTICULAR PURPOSE! 
%% User assumes all risk.
%% In no event shall IEEE or any contributor to this code be liable for
%% any damages or losses, including, but not limited to, incidental,
%% consequential, or any other damages, resulting from the use or misuse
%% of any information contained here.
%%
%% All comments are the opinions of their respective authors and are not
%% necessarily endorsed by the IEEE.
%%
%% This work is distributed under the LaTeX Project Public License (LPPL)
%% ( http://www.latex-project.org/ ) version 1.3, and may be freely used,
%% distributed and modified. A copy of the LPPL, version 1.3, is included
%% in the base LaTeX documentation of all distributions of LaTeX released
%% 2003/12/01 or later.
%% Retain all contribution notices and credits.
%% ** Modified files should be clearly indicated as such, including  **
%% ** renaming them and changing author support contact information. **
%%
%% File list of work: IEEEtran.cls, IEEEtran_HOWTO.pdf, bare_adv.tex,
%%                    bare_conf.tex, bare_jrnl.tex, bare_jrnl_compsoc.tex,
%%                    bare_jrnl_transmag.tex
%%*************************************************************************

% *** Authors should verify (and, if needed, correct) their LaTeX system  ***
% *** with the testflow diagnostic prior to trusting their LaTeX platform ***
% *** with production work. IEEE's font choices can trigger bugs that do  ***
% *** not appear when using other class files.                            ***
% The testflow support page is at:
% http://www.michaelshell.org/tex/testflow/



% Note that the a4paper option is mainly intended so that authors in
% countries using A4 can easily print to A4 and see how their papers will
% look in print - the typesetting of the document will not typically be
% affected with changes in paper size (but the bottom and side margins will).
% Use the testflow package mentioned above to verify correct handling of
% both paper sizes by the user's LaTeX system.
%
% Also note that the "draftcls" or "draftclsnofoot", not "draft", option
% should be used if it is desired that the figures are to be displayed in
% draft mode.
%
\documentclass[conference]{IEEEtran}
% Add the compsoc option for Computer Society conferences.
%
% If IEEEtran.cls has not been installed into the LaTeX system files,
% manually specify the path to it like:
% \documentclass[conference]{../sty/IEEEtran}





% Some very useful LaTeX packages include:
% (uncomment the ones you want to load)


% *** MISC UTILITY PACKAGES ***
%
%\usepackage{ifpdf}
% Heiko Oberdiek's ifpdf.sty is very useful if you need conditional
% compilation based on whether the output is pdf or dvi.
% usage:
% \ifpdf
%   % pdf code
% \else
%   % dvi code
% \fi
% The latest version of ifpdf.sty can be obtained from:
% http://www.ctan.org/tex-archive/macros/latex/contrib/oberdiek/
% Also, note that IEEEtran.cls V1.7 and later provides a builtin
% \ifCLASSINFOpdf conditional that works the same way.
% When switching from latex to pdflatex and vice-versa, the compiler may
% have to be run twice to clear warning/error messages.






% *** CITATION PACKAGES ***
%
%\usepackage{cite}
% cite.sty was written by Donald Arseneau
% V1.6 and later of IEEEtran pre-defines the format of the cite.sty package
% \cite{} output to follow that of IEEE. Loading the cite package will
% result in citation numbers being automatically sorted and properly
% "compressed/ranged". e.g., [1], [9], [2], [7], [5], [6] without using
% cite.sty will become [1], [2], [5]--[7], [9] using cite.sty. cite.sty's
% \cite will automatically add leading space, if needed. Use cite.sty's
% noadjust option (cite.sty V3.8 and later) if you want to turn this off
% such as if a citation ever needs to be enclosed in parenthesis.
% cite.sty is already installed on most LaTeX systems. Be sure and use
% version 4.0 (2003-05-27) and later if using hyperref.sty. cite.sty does
% not currently provide for hyperlinked citations.
% The latest version can be obtained at:
% http://www.ctan.org/tex-archive/macros/latex/contrib/cite/
% The documentation is contained in the cite.sty file itself.






% *** GRAPHICS RELATED PACKAGES ***
%
\ifCLASSINFOpdf
  % \usepackage[pdftex]{graphicx}
  % declare the path(s) where your graphic files are
  % \graphicspath{{../pdf/}{../jpeg/}}
  % and their extensions so you won't have to specify these with
  % every instance of \includegraphics
  % \DeclareGraphicsExtensions{.pdf,.jpeg,.png}
\else
  % or other class option (dvipsone, dvipdf, if not using dvips). graphicx
  % will default to the driver specified in the system graphics.cfg if no
  % driver is specified.
  % \usepackage[dvips]{graphicx}
  % declare the path(s) where your graphic files are
  % \graphicspath{{../eps/}}
  % and their extensions so you won't have to specify these with
  % every instance of \includegraphics
  % \DeclareGraphicsExtensions{.eps}
\fi
% graphicx was written by David Carlisle and Sebastian Rahtz. It is
% required if you want graphics, photos, etc. graphicx.sty is already
% installed on most LaTeX systems. The latest version and documentation
% can be obtained at: 
% http://www.ctan.org/tex-archive/macros/latex/required/graphics/
% Another good source of documentation is "Using Imported Graphics in
% LaTeX2e" by Keith Reckdahl which can be found at:
% http://www.ctan.org/tex-archive/info/epslatex/
%
% latex, and pdflatex in dvi mode, support graphics in encapsulated
% postscript (.eps) format. pdflatex in pdf mode supports graphics
% in .pdf, .jpeg, .png and .mps (metapost) formats. Users should ensure
% that all non-photo figures use a vector format (.eps, .pdf, .mps) and
% not a bitmapped formats (.jpeg, .png). IEEE frowns on bitmapped formats
% which can result in "jaggedy"/blurry rendering of lines and letters as
% well as large increases in file sizes.
%
% You can find documentation about the pdfTeX application at:
% http://www.tug.org/applications/pdftex





% *** MATH PACKAGES ***
%
\usepackage[cmex10]{amsmath}
% A popular package from the American Mathematical Society that provides
% many useful and powerful commands for dealing with mathematics. If using
% it, be sure to load this package with the cmex10 option to ensure that
% only type 1 fonts will utilized at all point sizes. Without this option,
% it is possible that some math symbols, particularly those within
% footnotes, will be rendered in bitmap form which will result in a
% document that can not be IEEE Xplore compliant!
%
% Also, note that the amsmath package sets \interdisplaylinepenalty to 10000
% thus preventing page breaks from occurring within multiline equations. Use:
%\interdisplaylinepenalty=2500
% after loading amsmath to restore such page breaks as IEEEtran.cls normally
% does. amsmath.sty is already installed on most LaTeX systems. The latest
% version and documentation can be obtained at:
% http://www.ctan.org/tex-archive/macros/latex/required/amslatex/math/





% *** SPECIALIZED LIST PACKAGES ***
%
%\usepackage{algorithmic}
% algorithmic.sty was written by Peter Williams and Rogerio Brito.
% This package provides an algorithmic environment fo describing algorithms.
% You can use the algorithmic environment in-text or within a figure
% environment to provide for a floating algorithm. Do NOT use the algorithm
% floating environment provided by algorithm.sty (by the same authors) or
% algorithm2e.sty (by Christophe Fiorio) as IEEE does not use dedicated
% algorithm float types and packages that provide these will not provide
% correct IEEE style captions. The latest version and documentation of
% algorithmic.sty can be obtained at:
% http://www.ctan.org/tex-archive/macros/latex/contrib/algorithms/
% There is also a support site at:
% http://algorithms.berlios.de/index.html
% Also of interest may be the (relatively newer and more customizable)
% algorithmicx.sty package by Szasz Janos:
% http://www.ctan.org/tex-archive/macros/latex/contrib/algorithmicx/




% *** ALIGNMENT PACKAGES ***
%
%\usepackage{array}
% Frank Mittelbach's and David Carlisle's array.sty patches and improves
% the standard LaTeX2e array and tabular environments to provide better
% appearance and additional user controls. As the default LaTeX2e table
% generation code is lacking to the point of almost being broken with
% respect to the quality of the end results, all users are strongly
% advised to use an enhanced (at the very least that provided by array.sty)
% set of table tools. array.sty is already installed on most systems. The
% latest version and documentation can be obtained at:
% http://www.ctan.org/tex-archive/macros/latex/required/tools/


% IEEEtran contains the IEEEeqnarray family of commands that can be used to
% generate multiline equations as well as matrices, tables, etc., of high
% quality.




% *** SUBFIGURE PACKAGES ***
%\ifCLASSOPTIONcompsoc
%  \usepackage[caption=false,font=normalsize,labelfont=sf,textfont=sf]{subfig}
%\else
%  \usepackage[caption=false,font=footnotesize]{subfig}
%\fi
% subfig.sty, written by Steven Douglas Cochran, is the modern replacement
% for subfigure.sty, the latter of which is no longer maintained and is
% incompatible with some LaTeX packages including fixltx2e. However,
% subfig.sty requires and automatically loads Axel Sommerfeldt's caption.sty
% which will override IEEEtran.cls' handling of captions and this will result
% in non-IEEE style figure/table captions. To prevent this problem, be sure
% and invoke subfig.sty's "caption=false" package option (available since
% subfig.sty version 1.3, 2005/06/28) as this is will preserve IEEEtran.cls
% handling of captions.
% Note that the Computer Society format requires a larger sans serif font
% than the serif footnote size font used in traditional IEEE formatting
% and thus the need to invoke different subfig.sty package options depending
% on whether compsoc mode has been enabled.
%
% The latest version and documentation of subfig.sty can be obtained at:
% http://www.ctan.org/tex-archive/macros/latex/contrib/subfig/




% *** FLOAT PACKAGES ***
%
%\usepackage{fixltx2e}
% fixltx2e, the successor to the earlier fix2col.sty, was written by
% Frank Mittelbach and David Carlisle. This package corrects a few problems
% in the LaTeX2e kernel, the most notable of which is that in current
% LaTeX2e releases, the ordering of single and double column floats is not
% guaranteed to be preserved. Thus, an unpatched LaTeX2e can allow a
% single column figure to be placed prior to an earlier double column
% figure. The latest version and documentation can be found at:
% http://www.ctan.org/tex-archive/macros/latex/base/


%\usepackage{stfloats}
% stfloats.sty was written by Sigitas Tolusis. This package gives LaTeX2e
% the ability to do double column floats at the bottom of the page as well
% as the top. (e.g., "\begin{figure*}[!b]" is not normally possible in
% LaTeX2e). It also provides a command:
%\fnbelowfloat
% to enable the placement of footnotes below bottom floats (the standard
% LaTeX2e kernel puts them above bottom floats). This is an invasive package
% which rewrites many portions of the LaTeX2e float routines. It may not work
% with other packages that modify the LaTeX2e float routines. The latest
% version and documentation can be obtained at:
% http://www.ctan.org/tex-archive/macros/latex/contrib/sttools/
% Do not use the stfloats baselinefloat ability as IEEE does not allow
% \baselineskip to stretch. Authors submitting work to the IEEE should note
% that IEEE rarely uses double column equations and that authors should try
% to avoid such use. Do not be tempted to use the cuted.sty or midfloat.sty
% packages (also by Sigitas Tolusis) as IEEE does not format its papers in
% such ways.
% Do not attempt to use stfloats with fixltx2e as they are incompatible.
% Instead, use Morten Hogholm'a dblfloatfix which combines the features
% of both fixltx2e and stfloats:
%
% \usepackage{dblfloatfix}
% The latest version can be found at:
% http://www.ctan.org/tex-archive/macros/latex/contrib/dblfloatfix/




% *** PDF, URL AND HYPERLINK PACKAGES ***
%
%\usepackage{url}
% url.sty was written by Donald Arseneau. It provides better support for
% handling and breaking URLs. url.sty is already installed on most LaTeX
% systems. The latest version and documentation can be obtained at:
% http://www.ctan.org/tex-archive/macros/latex/contrib/url/
% Basically, \url{my_url_here}.




% *** Do not adjust lengths that control margins, column widths, etc. ***
% *** Do not use packages that alter fonts (such as pslatex).         ***
% There should be no need to do such things with IEEEtran.cls V1.6 and later.
% (Unless specifically asked to do so by the journal or conference you plan
% to submit to, of course. )


% correct bad hyphenation here
\hyphenation{op-tical net-works semi-conduc-tor}


\begin{document}
%
% paper title
% can use linebreaks \\ within to get better formatting as desired
% Do not put math or special symbols in the title.
\title{Vulnerability of Wireless Fading Key Generation to Eavesdropping via Channel Prediction}


% author names and affiliations
% use a multiple column layout for up to three different
% affiliations
\author{\IEEEauthorblockN{Eric Brown}
\IEEEauthorblockA{Dept. of Electrical and\\Computer Engineering\\
University of Massachusettes Boston\\
Boston, MA (Zip code)\\
Email: Eric.Brown002@umb.edu}
\and
\IEEEauthorblockN{Clara Gamboa}
\IEEEauthorblockA{Dept. of Electrical and\\Computer Engineering\\
University of Massachusettes Boston\\
Boston, MA (Zip code)\\
Email: Clara.Gamboa001@umb.edu}
\and
\IEEEauthorblockN{Dr. K.C. Kerby-Patel}
\IEEEauthorblockA{Dept. of Electrical and\\Computer Engineering\\
University of Massachusettes Boston\\
Boston, MA (Zip code)\\
Email: KC.Kerby-Patel@umb.edu}}

% conference papers do not typically use \thanks and this command
% is locked out in conference mode. If really needed, such as for
% the acknowledgment of grants, issue a \IEEEoverridecommandlockouts
% after \documentclass

% for over three affiliations, or if they all won't fit within the width
% of the page, use this alternative format:
% 
%\author{\IEEEauthorblockN{Michael Shell\IEEEauthorrefmark{1},
%Homer Simpson\IEEEauthorrefmark{2},
%James Kirk\IEEEauthorrefmark{3}, 
%Montgomery Scott\IEEEauthorrefmark{3} and
%Eldon Tyrell\IEEEauthorrefmark{4}}
%\IEEEauthorblockA{\IEEEauthorrefmark{1}School of Electrical and Computer Engineering\\
%Georgia Institute of Technology,
%Atlanta, Georgia 30332--0250\\ Email: see http://www.michaelshell.org/contact.html}
%\IEEEauthorblockA{\IEEEauthorrefmark{2}Twentieth Century Fox, Springfield, USA\\
%Email: homer@thesimpsons.com}
%\IEEEauthorblockA{\IEEEauthorrefmark{3}Starfleet Academy, San Francisco, California 96678-2391\\
%Telephone: (800) 555--1212, Fax: (888) 555--1212}
%\IEEEauthorblockA{\IEEEauthorrefmark{4}Tyrell Inc., 123 Replicant Street, Los Angeles, California 90210--4321}}




% use for special paper notices
%\IEEEspecialpapernotice{(Invited Paper)}




% make the title area
\maketitle

% As a general rule, do not put math, special symbols or citations
% in the abstract
\begin{abstract}
Physical layer key generation techniques based on wireless channel fading are generally considered to be secure as long as any eavesdroppers are separated from the terminals by a distance greater than the channel correlation length.  This assertion depends on an assumption that the channel is ergodic, while many real-world channels are not ergodic.  Linear prediction has previously been employed to predict samples of the channel transfer function ahead in time.  In this work, we demonstrate that it is possible to predict the channel transfer function in space as well, over distances much larger than the correlation length.  %we will have to compute the correlation length from the observed channel covariance matrix.  
This leads us to conclude that a more pessimistic ``safe'' eavesdropper distance for wireless fading based key generation is needed.

\end{abstract}

\textbf{Key words:} -- wireless, channel, fading, key generation, physical layer, security




% For peer review papers, you can put extra information on the cover
% page as needed:
% \ifCLASSOPTIONpeerreview
% \begin{center} \bfseries EDICS Category: 3-BBND \end{center}
% \fi
%
% For peerreview papers, this IEEEtran command inserts a page break and
% creates the second title. It will be ignored for other modes.
\IEEEpeerreviewmaketitle



\section{Introduction}
% no \IEEEPARstart
It has been demonstrated \cite{anyPLkeygenpaper} that two parties' reciprocal observations of wireless channel fading can be used to generate symmetric encryption keys on the fly in a situation where two parties cannot pre-arrange keys. 
%possible applications
%references about the current state of this technique 
%security claims and their reasoning - implicit assumption of ergodicity by depending on the channel's correlation length
However, it has been shown that real-world wireless channels are typically not ergodic vs. time \cite{isukapalli}.  In this work, we show that the spatial average of such a channel's correlation function is not equal to the ensemble average, meaning that the channel is also not ergodic vs. space.  This indicates that security claims based on the correlation length are inappropriate.  To develop a more practical estimate of the minimum distance from a node to an eavesdropper, we apply channel prediction techniques \cite{any duel-hallen paper} to investigate the distance over which it is possible to predict the channel transfer function.

Section \ref{ergodicity} applies the method of \cite{isukapalli} to show that the spatial average of a channel's correlation function is not generally equal to the ensemble average, indicating that the channel is not spatially ergodic. Next, in Section \ref{prediction}, we apply long-range prediction techniques to a series of spatial samples in order to predict the channel transfer function over distances greater than the correlation length.  In Section \ref{simresults}, we apply this technique to a simple simulated channel to examine the dependence of prediction length on the properties of the channel and the eavesdropper array. %these are: number of scatterers; eavesdropper element spacing; and number of eavesdropper elements.

\section{Spatial Ergodicity} \label{ergodicity}
%discuss ergodicity and its significance for the security assumption here - why does this section exist?

This section follows the derivation by Isukapalli et al. \cite{isukapalli}, applying the same techniques to the spatial rather than temporal variation of the channel.

%added Doppler spatial phase variation to the channel model!!  f = fc+fd because f=(1+v/c)fc.  but k = 2pi/lambda, lambda = c/f, k = 2pi fc(1+v/c)/c.  so it's ok to write k as kc+kd, where kd = 2pi fd/c.
Begin by assuming that the channel is represented by a sum-of-sinusoids model as shown in Equation \ref{chan}.  The parameters $\alpha_n$, $\omega_d\cos\theta_n$, $k_d\cos\theta_n$, and $\vec{r_n}$ describe the amplitude, Doppler frequency shift, Doppler wavenumber, and location of the $n$th scatterer, and $\vec{r}$ represents the observation point.
\begin{equation}\label{chan}
h(\vec{r},t)= \sum_{n=1}^N \alpha_n e^{j \omega_d t \cos \theta_n} e^{-j(k+k_d\cos\theta_n)\left|\vec{r_n} - \vec{r}\right|}
\end{equation}
An observation of the channel impulse response at a new location, $\vec{r}+\vec{\Delta r}$, can be written as shown in Equation \ref{chanloc2} if $|\vec{\Delta r}|$ is small enough that all scatterers are in the far field of the pair of observation points.
\begin{equation}\label{chanloc2}
h(\vec{r}+\vec{\Delta r},t) = \sum_{m=1}^N \alpha_m e^{j \omega_d t \cos \theta_m} e^{-j(k+k_d\cos\theta_n)\left|\vec{r_m}-\vec{r}\right|}e^{-jk\left|\vec{\Delta r}\right| \cos \psi_m}
\end{equation}

The correlation function (based on the ensemble average) for two spatially separated but simultaneous observations can be separated into a single sum of squared terms from the same scatterer (the first term in Equation \ref{ensemblecorr}) and a double sum of contributions from two different scatterers (the second term).
\begin{multline}\label{ensemblecorr}
R_e(\vec{\Delta r})= E\left[\sum_{n=m=1}^N |\alpha_n|^2 e^{-j(k+k_d\cos\theta_n)|\vec{\Delta r}|\cos \psi_n}\right] +\\ E\left[\sum_{n=1}^N\sum_{m\neq n} \alpha_n^* \alpha_m e^{j \omega_d (\cos \theta_m - \cos \theta_n)} e^{-j(k+k_d\cos\theta_n)\left(\left|\vec{r_m} - \vec{r}\right|-\left|\vec{r_n} - \vec{r}\right|\right)}e^{-jk\left|\vec{\Delta r}\right| \cos \psi_m}\right]
\end{multline}
The second expected value term in Equation \ref{ensemblecorr} is eliminated by the integration of the complex exponential containing $\left|\vec{r_m} - \vec{r}\right|-\left|\vec{r_n} - \vec{r}\right|$ during calculation of the expected value.  

To simplify further, we assume that $E[|\alpha|^2]=1/N$.  Now the correlation function depends on the distributions of $\theta$ and $\psi$:
\begin{equation}\label{generalcorr}
R_e(\vec{\Delta r})=\iint_{-\pi, -\pi}^{\pi,\pi}e^{-j k |\vec{\Delta r}| \cos(u)} e^{-j k_d |\vec{\Delta r}| \cos (v) \cos(u)} p_{\theta}(v) p_{\psi}(u) du dv
\end{equation}

Equation \ref{generalcorr} is simplified as much as is possible without making any assumptions about the probability distributions of $\theta$ and $\psi$.  However, if $\theta$ is uniformly distributed, this can be simplified to 
\begin{equation}
R_e(\vec{\Delta r})=\int_{-\pi}^{\pi}e^{-j k |\vec{\Delta r}| \cos(u)} J_0(k_d |\vec{\Delta r}| \cos(u))p_{\psi}(u) du 
\end{equation}
%the math departs from Isukapalli here because of the addition of the Doppler wavenumber. 

If $\psi$ is uniformly distributed, we can use the identity in Equation \ref{identity} \cite{integraltablebook} and a plane wave expansion to evaluate the integral with respect to $u$, arriving at Equation \ref{finalcorr}.
\begin{equation}\label{identity}
\int_0^{\pi} J_0(2z\cos(x))cox(2nx)dx = (-1)^n \pi J_n^2(z)
\end{equation}
% Gradshteyn, I. and Ryzhik, I., Table of Integrals, Series, and Products, 7th edition, Academic Press: Amsterdam, 2007 page 724
\begin{equation}\label{finalcorr}
R_e(\vec{\Delta_r}) = J_0^2(\frac{k_d |\vec{\Delta_r}|}{2})J_0(k|\vec{\Delta_r}|) + 2\sum_{n=1}^{\infty}(-1)^n J_n(k|\vec{\Delta_r}|)J_{n/2}^2(\frac{k_d |\vec{\Delta_r}|}{2})
\end{equation}

On the other hand, the spatial average of $h(\vec{r}+\vec{\Delta_r},t)h^*(\vec{r},t)$ is given by
\begin{multline}
R_r\left(\vec{\Delta r}\right) =  \lim_{D \rightarrow \infty} \frac{3}{4\pi D^3}\int_0^D\int_{-\pi}^\pi \int_{-\pi}^{\pi} \sum_{n=m=1}^N |\alpha_n|^2 e^{-j(k+k_d\cos\theta_n)|\vec{\Delta r}|\cos \psi_n}+\\ \sum_{n=1}^N\sum_{m\neq n} \alpha_n^* \alpha_m e^{j \omega_d t(\cos \theta_m - \cos \theta_n)} e^{-j(k+k_d\cos\theta_n)\left(\left|\vec{r_m} - \vec{r}\right|-\left|\vec{r_n} - \vec{r}\right|\right)}e^{-jk\left|\vec{\Delta r}\right| \cos \psi_m}r^2 \sin\theta dr d\theta d\phi
\end{multline}
Again the second term is eliminated by the integration of the complex exponential containing $\left|\vec{r_m} - \vec{r}\right|-\left|\vec{r_n} - \vec{r}\right|$.  What remains in the integrand does not depend on position, so the spatial average correlation is given by 
\begin{equation}\label{spatialcorr}
R_r\left(\vec{\Delta r}\right) =  \sum_{n=1}^N |\alpha_n|^2 e^{-j(k+k_d\cos\theta_n)|\vec{\Delta r}|\cos \psi_n}
\end{equation}

Depending on the number of scatterers $N$ and their angular distribution, Equation \ref{spatialcorr} may approach Equation \ref{finalcorr} \cite{isukapalli}.  However, real-world channels typically do not have sufficient scatterers to achieve this condition \cite{was it Duel-Hallen who said this?}.  
%physically, no new realization of the channel occurs over the time or spatial interval in which we observe it - we are still in the same channel realization
% in this case, successive temporal or spatial samples do contain some information about each other
% we can use a series of samples to predict the channel transfer function at other temporal or spatial locations (cite duel-hallen)

%this section might need some graphs in it, demonstrating how the correlation function and the spatial sample covariance are different for different situations

\section{Spatial Long-Range Channel Prediction Method}\label{prediction}
%since the channel's spatial variation is not ergodic, it is well suited to long-range prediction techniques (cite duel-hallen). rather than using long-range prediction to predict the channel transfer function ahead temporally, we predict it spatially.
%describe the problem setup: channel, eavesdropper
%assumption: the angle from eavesdropper elements to scatterers stays the same over all the eavesdropper samples and predicted samples (all scatterers are in the far field of an array of length Nd+Qd)

We are doing things differently %(differently from what? we have to cite so people know what to compare to). 

We take the channel model that is usually a fuction of time and align it along a listening array and transform it into a function of space. (This needs to be explained as the mathematical equations are given. Describe the variables and whatnot)

\section{Spatial Long-Range Channel Prediction Simulated Results}\label{simresults}
%in this section we apply the method from the previous section to a simulated channel
%describe the channel simulation (very similar to problem setup from previous section)
%find average prediction length and how it varies for different values of $d$ (element spacing in E), $N$ (number of elements in E), and $S$ (number of scatterers)

In this section we will go on to describe how we set about creating a channel model and the methods we used. (LPC, ARYule, MATLAB code explanations and code may be used here.) This section should have graphs that show our channel model and explain the methodology clearly. We are using the sum of sinusoids method to assemble a wireless channel with $S$ scatterers, $N$ number of array elements (spatial samples) etc... In this section we also discuss how we went about our predictions and how far out we can reasonably predict. How many times were the tests run?



% An example of a floating figure using the graphicx package.
% Note that \label must occur AFTER (or within) \caption.
% For figures, \caption should occur after the \includegraphics.
% Note that IEEEtran v1.7 and later has special internal code that
% is designed to preserve the operation of \label within \caption
% even when the captionsoff option is in effect. However, because
% of issues like this, it may be the safest practice to put all your
% \label just after \caption rather than within \caption{}.
%
% Reminder: the "draftcls" or "draftclsnofoot", not "draft", class
% option should be used if it is desired that the figures are to be
% displayed while in draft mode.
%
%\begin{figure}[!t]
%\centering
%\includegraphics[width=2.5in]{myfigure}
% where an .eps filename suffix will be assumed under latex, 
% and a .pdf suffix will be assumed for pdflatex; or what has been declared
% via \DeclareGraphicsExtensions.
%\caption{Simulation Results.}
%\label{fig_sim}
%\end{figure}

% Note that IEEE typically puts floats only at the top, even when this
% results in a large percentage of a column being occupied by floats.


% An example of a double column floating figure using two subfigures.
% (The subfig.sty package must be loaded for this to work.)
% The subfigure \label commands are set within each subfloat command,
% and the \label for the overall figure must come after \caption.
% \hfil is used as a separator to get equal spacing.
% Watch out that the combined width of all the subfigures on a 
% line do not exceed the text width or a line break will occur.
%
%\begin{figure*}[!t]
%\centering
%\subfloat[Case I]{\includegraphics[width=2.5in]{box}%
%\label{fig_first_case}}
%\hfil
%\subfloat[Case II]{\includegraphics[width=2.5in]{box}%
%\label{fig_second_case}}
%\caption{Simulation results.}
%\label{fig_sim}
%\end{figure*}
%
% Note that often IEEE papers with subfigures do not employ subfigure
% captions (using the optional argument to \subfloat[]), but instead will
% reference/describe all of them (a), (b), etc., within the main caption.


% An example of a floating table. Note that, for IEEE style tables, the 
% \caption command should come BEFORE the table. Table text will default to
% \footnotesize as IEEE normally uses this smaller font for tables.
% The \label must come after \caption as always.
%
%\begin{table}[!t]
%% increase table row spacing, adjust to taste
%\renewcommand{\arraystretch}{1.3}
% if using array.sty, it might be a good idea to tweak the value of
% \extrarowheight as needed to properly center the text within the cells
%\caption{An Example of a Table}
%\label{table_example}
%\centering
%% Some packages, such as MDW tools, offer better commands for making tables
%% than the plain LaTeX2e tabular which is used here.
%\begin{tabular}{|c||c|}
%\hline
%One & Two\\
%\hline
%Three & Four\\
%\hline
%\end{tabular}
%\end{table}


% Note that IEEE does not put floats in the very first column - or typically
% anywhere on the first page for that matter. Also, in-text middle ("here")
% positioning is not used. Most IEEE journals/conferences use top floats
% exclusively. Note that, LaTeX2e, unlike IEEE journals/conferences, places
% footnotes above bottom floats. This can be corrected via the \fnbelowfloat
% command of the stfloats package.



\section{Conclusion}
%we applied long-range prediction to estimate the channel transfer function at a location in space that is separated from the eavesdropper by more than a correlation length.  This indicates that the minimum safe distance for secure key generation based on wireless fading must be reexamined.

<<<<<<< HEAD
%future work: do a scenario where the eavesdropper moves through space and see if we can do spatiotemporal channel prediction simultaneously, see about 2D eavesdropper arrays and other problem geometries (what if all the scatterers are in an arc of the space instead of evenly distributed? does it matter?), do a security analysis by estimating the error of the prediction(?), demonstrate spatial channel prediction experimentally, investigate the effect of mutual coupling between array elements on the quality of the eavesdropper's predictions




% conference papers do not normally have an appendix


% use section* for acknowledgement
\section*{Acknowledgment}


The authors would like to thank...





% trigger a \newpage just before the given reference
% number - used to balance the columns on the last page
% adjust value as needed - may need to be readjusted if
% the document is modified later
%\IEEEtriggeratref{8}
% The "triggered" command can be changed if desired:
%\IEEEtriggercmd{\enlargethispage{-5in}}

% references section

% can use a bibliography generated by BibTeX as a .bbl file
% BibTeX documentation can be easily obtained at:
% http://www.ctan.org/tex-archive/biblio/bibtex/contrib/doc/
% The IEEEtran BibTeX style support page is at:
% http://www.michaelshell.org/tex/ieeetran/bibtex/
%\bibliographystyle{IEEEtran}
% argument is your BibTeX string definitions and bibliography database(s)
%\bibliography{IEEEabrv,../bib/paper}
%
% <OR> manually copy in the resultant .bbl file
% set second argument of \begin to the number of references
% (used to reserve space for the reference number labels box)
\begin{thebibliography}{1}

\bibitem{IEEEhowto:kopka}
H.~Kopka and P.~W. Daly, \emph{A Guide to \LaTeX}, 3rd~ed.\hskip 1em plus
  0.5em minus 0.4em\relax Harlow, England: Addison-Wesley, 1999.

\end{thebibliography}




% that's all folks
=======
%future work: add Doppler dependence, do a scenario where the eavesdropper moves through space and see if we can do spatiotemporal channel prediction simultaneously, see about 2D eavesdropper arrays and other problem geometries (what if all the scatterers are in an arc of the space instead of evenly distributed? does it matter?), do a security analysis by estimating the error of the prediction(?), demonstrate spatial channel prediction experimentally, investigate the effect of mutual coupling between array elements on the quality of the eavesdropper's predictions
>>>>>>> parent of 143cb61... Added *.asv to .gitignore
\end{document}


