% !TEX TS-program = pdflatex
% !TEX encoding = UTF-8 Unicode

% This is a simple template for a LaTeX document using the "article" class.
% See "book", "report", "letter" for other types of document.

\documentclass[11pt]{article} % This will be \documentclass{ieeetran} later

\usepackage[utf8]{inputenc} % set input encoding (not needed with XeLaTeX)

%%% Examples of Article customizations
% These packages are optional, depending whether you want the features they provide.
% See the LaTeX Companion or other references for full information.

%%% PAGE DIMENSIONS
\usepackage{geometry} % to change the page dimensions
\geometry{a4paper} % or letterpaper (US) or a5paper or....
% \geometry{margin=2in} % for example, change the margins to 2 inches all round
% \geometry{landscape} % set up the page for landscape
%   read geometry.pdf for detailed page layout information

\usepackage{graphicx} % support the \includegraphics command and options
\usepackage{amsmath}
% \usepackage[parfill]{parskip} % Activate to begin paragraphs with an empty line rather than an indent

%%% PACKAGES
\usepackage{booktabs} % for much better looking tables
\usepackage{array} % for better arrays (eg matrices) in maths
\usepackage{paralist} % very flexible & customisable lists (eg. enumerate/itemize, etc.)
\usepackage{verbatim} % adds environment for commenting out blocks of text & for better verbatim
\usepackage{subfig} % make it possible to include more than one captioned figure/table in a single float
% These packages are all incorporated in the memoir class to one degree or another...

%%% HEADERS & FOOTERS
\usepackage{fancyhdr} % This should be set AFTER setting up the page geometry
\pagestyle{fancy} % options: empty , plain , fancy
\renewcommand{\headrulewidth}{0pt} % customise the layout...
\lhead{}\chead{}\rhead{}
\lfoot{}\cfoot{\thepage}\rfoot{}

%%% SECTION TITLE APPEARANCE
\usepackage{sectsty}
\allsectionsfont{\sffamily\mdseries\upshape} % (See the fntguide.pdf for font help)
% (This matches ConTeXt defaults)

%%% ToC (table of contents) APPEARANCE
\usepackage[nottoc,notlof,notlot]{tocbibind} % Put the bibliography in the ToC
\usepackage[titles,subfigure]{tocloft} % Alter the style of the Table of Contents
\renewcommand{\cftsecfont}{\rmfamily\mdseries\upshape}
\renewcommand{\cftsecpagefont}{\rmfamily\mdseries\upshape} % No bold!

%%% END Article customizations

%%% The "real" document content comes below...

\title{Vulnerability of Wireless Fading Key Generation to Eavesdropping via Channel Prediction} %this title is not bad
\author{Clara Gamboa, Eric Brown, K.C. Kerby-Patel}  %the convention is that the advisor's name goes last
%also have to put in our UMB affiliation blurb
%\date{} % Activate to display a given date or no date (if empty),
         % otherwise the current date is printed 

\begin{document}
\maketitle

\section{Abstract}

Physical layer key generation techniques based on wireless channel fading are generally considered to be secure as long as any eavesdroppers are separated from the terminals by a distance greater than the channel correlation length.  This assertion depends on an assumption that the channel is ergodic, while many real-world channels are not ergodic.  Linear prediction has previously been employed to predict samples of the channel transfer function ahead in time.  In this work, we demonstrate that it is possible to predict the channel transfer function in space as well, over distances much larger than the correlation length.  %we will have to compute the correlation length from the observed channel covariance matrix.  
This leads us to conclude that a more pessimistic ``safe'' eavesdropper distance for wireless fading based key generation is needed.

\textbf{Key words:} -- wireless, channel, fading, key generation, physical layer, security

\section{Introduction}

It has been demonstrated \cite{anyPLkeygenpaper} that two parties' reciprocal observations of wireless channel fading can be used to generate symmetric encryption keys on the fly in a situation where two parties cannot pre-arrange keys. 
%possible applications
%references about the current state of this technique 
%security claims and their reasoning - implicit assumption of ergodicity by depending on the channel's correlation length
However, it has been shown that real-world wireless channels are typically not ergodic vs. time \cite{isukapalli}.  In this work, we show that the spatial average of such a channel's correlation function is not equal to the ensemble average, meaning that the channel is also not ergodic vs. space.  This indicates that security claims based on the correlation length are inappropriate.  To develop a more practical estimate of the minimum distance from a node to an eavesdropper, we apply channel prediction techniques \cite{any duel-hallen paper} to investigate the distance over which it is possible to predict the channel transfer function.

Section \ref{ergodicity} applies the method of \cite{isukapalli} to show that the spatial average of a channel's correlation function is not generally equal to the ensemble average, indicating that the channel is not spatially ergodic. Next, in Section \ref{prediction}, we apply long-range prediction techniques to a series of spatial samples in order to predict the channel transfer function over distances greater than the correlation length.  In Section \ref{simresults}, we apply this technique to a simple simulated channel to examine the dependence of prediction length on the properties of the channel and the eavesdropper array. %these are: number of scatterers; eavesdropper element spacing; and number of eavesdropper elements.


\section{Spatial Ergodicity}\label{ergodicity}

%discuss ergodicity and its significance for the security assumption here - why does this section exist?

This section follows the derivation by Isukapalli et al. \cite{isukapalli}, applying the same techniques to the spatial rather than temporal variation of the channel.

%added Doppler spatial phase variation to the channel model!!  f = fc+fd because f=(1+v/c)fc.  but k = 2pi/lambda, lambda = c/f, k = 2pi fc(1+v/c)/c.  so it's ok to write k as kc+kd, where kd = 2pi fd/c.
Begin by assuming that the channel is represented by a sum-of-sinusoids model as shown in Equation \ref{chan}.  The parameters $\alpha_n$, $\omega_d\cos\theta_n$, $k_d\cos\theta_n$, and $\vec{r_n}$ describe the amplitude, Doppler frequency shift, Doppler wavenumber, and location of the $n$th scatterer, and $\vec{r}$ represents the observation point.
\begin{equation}\label{chan}
h(\vec{r},t)= \sum_{n=1}^N \alpha_n e^{j \omega_d t \cos \theta_n} e^{-j(k+k_d\cos\theta_n)\left|\vec{r_n} - \vec{r}\right|}
\end{equation}
An observation of the channel impulse response at a new location, $\vec{r}+\vec{\Delta r}$, can be written as shown in Equation \ref{chanloc2} if $|\vec{\Delta r}|$ is small enough that all scatterers are in the far field of the pair of observation points.
\begin{equation}\label{chanloc2}
h(\vec{r}+\vec{\Delta r},t) = \sum_{m=1}^N \alpha_m e^{j \omega_d t \cos \theta_m} e^{-j(k+k_d\cos\theta_n)\left|\vec{r_m}-\vec{r}\right|}e^{-jk\left|\vec{\Delta r}\right| \cos \psi_m}
\end{equation}

The correlation function (based on the ensemble average) for two spatially separated but simultaneous observations can be separated into a single sum of squared terms from the same scatterer (the first term in Equation \ref{ensemblecorr}) and a double sum of contributions from two different scatterers (the second term).
\begin{multline}\label{ensemblecorr}
R_e(\vec{\Delta r})= E\left[\sum_{n=m=1}^N |\alpha_n|^2 e^{-j(k+k_d\cos\theta_n)|\vec{\Delta r}|\cos \psi_n}\right] +\\ E\left[\sum_{n=1}^N\sum_{m\neq n} \alpha_n^* \alpha_m e^{j \omega_d (\cos \theta_m - \cos \theta_n)} e^{-j(k+k_d\cos\theta_n)\left(\left|\vec{r_m} - \vec{r}\right|-\left|\vec{r_n} - \vec{r}\right|\right)}e^{-jk\left|\vec{\Delta r}\right| \cos \psi_m}\right]
\end{multline}
The second expected value term in Equation \ref{ensemblecorr} is eliminated by the integration of the complex exponential containing $\left|\vec{r_m} - \vec{r}\right|-\left|\vec{r_n} - \vec{r}\right|$ during calculation of the expected value.  

To simplify further, we assume that $E[|\alpha|^2]=1/N$.  Now the correlation function depends on the distributions of $\theta$ and $\psi$:
\begin{equation}\label{generalcorr}
R_e(\vec{\Delta r})=\iint_{-\pi, -\pi}^{\pi,\pi}e^{-j k |\vec{\Delta r}| \cos(u)} e^{-j k_d |\vec{\Delta r}| \cos (v) \cos(u)} p_{\theta}(v) p_{\psi}(u) du dv
\end{equation}

Equation \ref{generalcorr} is simplified as much as is possible without making any assumptions about the probability distributions of $\theta$ and $\psi$.  However, if $\theta$ is uniformly distributed, this can be simplified to 
\begin{equation}
R_e(\vec{\Delta r})=\int_{-\pi}^{\pi}e^{-j k |\vec{\Delta r}| \cos(u)} J_0(k_d |\vec{\Delta r}| \cos(u))p_{\psi}(u) du 
\end{equation}
%the math departs from Isukapalli here because of the addition of the Doppler wavenumber. 

If $\psi$ is uniformly distributed, we can use the identity in Equation \ref{identity} \cite{integraltablebook} and a plane wave expansion to evaluate the integral with respect to $u$, arriving at Equation \ref{finalcorr}.
\begin{equation}\label{identity}
\int_0^{\pi} J_0(2z\cos(x))cox(2nx)dx = (-1)^n \pi J_n^2(z)
\end{equation}
% Gradshteyn, I. and Ryzhik, I., Table of Integrals, Series, and Products, 7th edition, Academic Press: Amsterdam, 2007 page 724
\begin{equation}\label{finalcorr}
R_e(\vec{\Delta_r}) = J_0^2(\frac{k_d |\vec{\Delta_r}|}{2})J_0(k|\vec{\Delta_r}|) + 2\sum_{n=1}^{\infty}(-1)^n J_n(k|\vec{\Delta_r}|)J_{n/2}^2(\frac{k_d |\vec{\Delta_r}|}{2})
\end{equation}

On the other hand, the spatial average of $h(\vec{r}+\vec{\Delta_r},t)h^*(\vec{r},t)$ is given by
\begin{multline}
R_r\left(\vec{\Delta r}\right) =  \lim_{D \rightarrow \infty} \frac{3}{4\pi D^3}\int_0^D\int_{-\pi}^\pi \int_{-\pi}^{\pi} \sum_{n=m=1}^N |\alpha_n|^2 e^{-j(k+k_d\cos\theta_n)|\vec{\Delta r}|\cos \psi_n}+\\ \sum_{n=1}^N\sum_{m\neq n} \alpha_n^* \alpha_m e^{j \omega_d t(\cos \theta_m - \cos \theta_n)} e^{-j(k+k_d\cos\theta_n)\left(\left|\vec{r_m} - \vec{r}\right|-\left|\vec{r_n} - \vec{r}\right|\right)}e^{-jk\left|\vec{\Delta r}\right| \cos \psi_m}r^2 \sin\theta dr d\theta d\phi
\end{multline}
Again the second term is eliminated by the integration of the complex exponential containing $\left|\vec{r_m} - \vec{r}\right|-\left|\vec{r_n} - \vec{r}\right|$.  What remains in the integrand does not depend on position, so the spatial average correlation is given by 
\begin{equation}\label{spatialcorr}
R_r\left(\vec{\Delta r}\right) =  \sum_{n=1}^N |\alpha_n|^2 e^{-j(k+k_d\cos\theta_n)|\vec{\Delta r}|\cos \psi_n}
\end{equation}

Depending on the number of scatterers $N$ and their angular distribution, Equation \ref{spatialcorr} may approach Equation \ref{finalcorr} \cite{isukapalli}.  However, real-world channels typically do not have sufficient scatterers to achieve this condition \cite{was it Duel-Hallen who said this?}.  
%physically, no new realization of the channel occurs over the time or spatial interval in which we observe it - we are still in the same channel realization
% in this case, successive temporal or spatial samples do contain some information about each other
% we can use a series of samples to predict the channel transfer function at other temporal or spatial locations (cite duel-hallen)

%this section might need some graphs in it, demonstrating how the correlation function and the spatial sample covariance are different for different situations

\section{Spatial Long-Range Channel Prediction Method}\label{prediction}
%since the channel's spatial variation is not ergodic, it is well suited to long-range prediction techniques (cite duel-hallen). rather than using long-range prediction to predict the channel transfer function ahead temporally, we predict it spatially.
%describe the problem setup: channel, eavesdropper
%assumption: the angle from eavesdropper elements to scatterers stays the same over all the eavesdropper samples and predicted samples (all scatterers are in the far field of an array of length Nd+Qd)

We are doing things differently %(differently from what? we have to cite so people know what to compare to). 

We take the channel model that is usually a fuction of time and align it along a listening array and transform it into a function of space. (This needs to be explained as the mathematical equations are given. Describe the variables and whatnot)

\section{Spatial Long-Range Channel Prediction Simulated Results}\label{simresults}
%in this section we apply the method from the previous section to a simulated channel
%describe the channel simulation (very similar to problem setup from previous section)
%find average prediction length and how it varies for different values of $d$ (element spacing in E), $N$ (number of elements in E), and $S$ (number of scatterers)

In this section we will go on to describe how we set about creating a channel model and the methods we used. (LPC, ARYule, MATLAB code explanations and code may be used here.) This section should have graphs that show our channel model and explain the methodology clearly. We are using the sum of sinusoids method to assemble a wireless channel with $S$ scatterers, $N$ number of array elements (spatial samples) etc... In this section we also discuss how we went about our predictions and how far out we can reasonably predict. How many times were the tests run?

\section{Conclusion}
%we applied long-range prediction to estimate the channel transfer function at a location in space that is separated from the eavesdropper by more than a correlation length.  This indicates that the minimum safe distance for secure key generation based on wireless fading must be reexamined.

%future work: add Doppler dependence, do a scenario where the eavesdropper moves through space and see if we can do spatiotemporal channel prediction simultaneously, see about 2D eavesdropper arrays and other problem geometries (what if all the scatterers are in an arc of the space instead of evenly distributed? does it matter?), do a security analysis by estimating the error of the prediction(?), demonstrate spatial channel prediction experimentally, investigate the effect of mutual coupling between array elements on the quality of the eavesdropper's predictions
\end{document}
