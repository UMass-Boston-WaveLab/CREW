\documentclass{allertonproc}
\usepackage{amsmath}
\usepackage{esint}
\usepackage{graphicx}
\usepackage{amsmath}
\usepackage{cite}


\title{SPATIAL CHANNEL PREDICTION FOR EAVESDROPPING ON WIRELESS FADING BASED KEY GENERATION}
\date{}
\author{Eric Brown, Clara Gamboa, and K.~C.~Kerby-Patel\\University of Massachusetts Boston, Boston, MA\\kc.kerby-patel@umb.edu}
\begin{document}
\maketitle
<<<<<<< Updated upstream
\abstract{
%people try to generate symmetric keys using the inherent reciprocity of the wireless channel
%it's generally stated that the 
}
=======
\begin{abstract}

Physical layer key generation techniques based on wireless channel fading are generally considered to be secure as long as any eavesdroppers are separated from the terminals by a distance greater than the channel correlation length.  This assertion depends on an assumption that the channel is ergodic, while many real-world channels are not ergodic.  Since a non-ergodic channel's transfer function depends deterministically on the physical scatterers in that particular channel realization, it is possible to predict the channel transfer function in space using techniques similar to those proposed for long-range temporal channel prediction.  We demonstrate the application of long-range prediction techniques to a simple sum-of-sinusoids channel model.  By estimating the spatial frequencies and amplitudes associated with the scattering paths, the channel transfer function can be predicted over distances larger than the correlation length.  This leads to the conclusion that a more pessimistic ``safe'' eavesdropper distance for wireless fading based key generation is needed.
\end{abstract}
>>>>>>> Stashed changes
\end{document}