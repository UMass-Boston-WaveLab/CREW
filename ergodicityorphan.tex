\documentclass{allertonproc} 

%%% Examples of Article customizations
% These packages are optional, depending whether you want the features they provide.
% See the LaTeX Companion or other references for full information.
\usepackage{cite}
\usepackage{graphicx} % support the \includegraphics command and options
\usepackage{amsmath}
% \usepackage[parfill]{parskip} % Activate to begin paragraphs with an empty line rather than an indent

%%% PACKAGES
\usepackage{booktabs} % for much better looking tables
\usepackage{array} % for better arrays (eg matrices) in maths
\usepackage{paralist} % very flexible & customisable lists (eg. enumerate/itemize, etc.)
\usepackage{verbatim} % adds environment for commenting out blocks of text & for better verbatim
\usepackage{subfig} % make it possible to include more than one captioned figure/table in a single float
% These packages are all incorporated in the memoir class to one degree or another...

\title{Orphaned Ergodicity Math}
\date{July 6 2015}
\author{K.~C.~Kerby-Patel}
\begin{document}
\maketitle

Past work on key generation from wireless fading has often stated that the key thus generated can be considered secure as long as all eavesdroppers are separated by at least half a carrier wavelength from both of the communicating nodes \cite{azimisadjadi2007, bloch2008, mathur2008, ye2010}.  This is based on the fact that the channel correlation function drops off quickly as a function of distance \cite{jakes1974}.  However, low correlation does not necessarily indicate low mutual information.  The correlation function is proportional to the mutual information if the two observations are jointly Gaussian \cite{li1990}; however, this is only the case if the channel is ergodic.  It has been shown by Isukapalli et al. \cite{isukapalli2006} that a typical wireless channel is not ergodic in the temporal sense (that is, the time average of the correlation function is not necessarily equal to its ensemble average).  In this section, we will follow the derivation from \cite{isukapalli2006} to show that typical wireless channels are also not spatially ergodic (that is, the spatial average of the correlation function is not necessarily equal to its ensemble average), as long as the total sampled length is sufficiently short.

%added Doppler spatial phase variation to the channel model!!  f = fc+fd because f=(1+v/c)fc.  but k = 2pi/lambda, lambda = c/f, k = 2pi fc(1+v/c)/c.  so it's ok to write k as kc+kd, where kd = 2pi fd/c.
Begin by assuming that the channel is represented by a sum-of-sinusoids model as shown in Equation \ref{chan}.  The parameters $\alpha_n$, $\omega_d\cos\theta_n$, $k_d\cos\theta_n$, and $\vec{r_n}$ describe the amplitude, Doppler frequency shift, Doppler wavenumber, and location of the $n$th scatterer, and $\vec{r}$ represents the observation point.
\begin{equation}\label{chan}
h(\vec{r},t)= \sum_{n=1}^N \alpha_n e^{j \omega_d t \cos \theta_n} e^{-j(k+k_d\cos\theta_n)\left|\vec{r_n} - \vec{r}\right|}
\end{equation}
An observation of the channel impulse response at a new location, $\vec{r}+\vec{\Delta r}$, can be written as shown in Equation \ref{chanloc2} if $|\vec{\Delta r}|$ is small enough that all scatterers are in the far field of the pair of observation points.
\begin{equation}\label{chanloc2}
h(\vec{r}+\vec{\Delta r},t) = \sum_{n=1}^N \alpha_n e^{j \omega_d t \cos \theta_n} e^{-j(k+k_d\cos\theta_n)\left|\vec{r_n}-\vec{r}\right|}e^{-jk\left|\vec{\Delta r}\right| \cos \psi_n}
\end{equation}

The correlation function (based on the ensemble average) for two spatially separated but simultaneous observations can be separated into a single sum of squared terms from the same scatterer (the first term in Equation \ref{ensemblecorr}) and a double sum of contributions from two different scatterers (the second term).
\begin{multline}\label{ensemblecorr}
R_e(\vec{\Delta r})= E\left[\sum_{n=m=1}^N |\alpha_n|^2 e^{-j(k+k_d\cos\theta_n)|\vec{\Delta r}|\cos \psi_n}\right] +\\ E\left[\sum_{n=1}^N\sum_{m\neq n} \alpha_n^* \alpha_m e^{j \omega_d (\cos \theta_m - \cos \theta_n)} e^{-j(k+k_d\cos\theta_n)\left(\left|\vec{r_m} - \vec{r}\right|-\left|\vec{r_n} - \vec{r}\right|\right)}e^{-jk\left|\vec{\Delta r}\right| \cos \psi_m}\right]
\end{multline}
The second expected value term in Equation \ref{ensemblecorr} is eliminated by the integration of the complex exponential containing $\left|\vec{r_m} - \vec{r}\right|-\left|\vec{r_n} - \vec{r}\right|$ during calculation of the expected value.  

To simplify further, we assume that $E[|\alpha|^2]=1/N$.  Now the correlation function depends on the distributions of $\theta$ and $\psi$:
\begin{equation}\label{generalcorr}
R_e(\vec{\Delta r})=\iint_{-\pi, -\pi}^{\pi,\pi}e^{-j k |\vec{\Delta r}| \cos(u)} e^{-j k_d |\vec{\Delta r}| \cos (v) \cos(u)} p_{\theta}(v) p_{\psi}(u) du dv
\end{equation}

Equation \ref{generalcorr} is simplified as much as is possible without making any assumptions about the probability distributions of $\theta$ and $\psi$.  However, if $\theta$ is uniformly distributed, this can be simplified to 
\begin{equation}
R_e(\vec{\Delta r})=\int_{-\pi}^{\pi}e^{-j k |\vec{\Delta r}| \cos(u)} J_0(k_d |\vec{\Delta r}| \cos(u))p_{\psi}(u) du 
\end{equation}
%the math departs from Isukapalli here because of the addition of the Doppler wavenumber. 

If $\psi$ is uniformly distributed, we can use the identity in Equation \ref{identity} \cite{gradshteyn2007} and a plane wave expansion to evaluate the integral with respect to $u$, arriving at Equation \ref{finalcorr}.
\begin{equation}\label{identity}
\int_0^{\pi} J_0(2z\cos(x))cox(2nx)dx = (-1)^n \pi J_n^2(z)
\end{equation}
% Gradshteyn, I. and Ryzhik, I., Table of Integrals, Series, and Products, 7th edition, Academic Press: Amsterdam, 2007 page 724
\begin{equation}\label{finalcorr}
R_e(\vec{\Delta_r}) = J_0^2(\frac{k_d |\vec{\Delta_r}|}{2})J_0(k|\vec{\Delta_r}|) + 2\sum_{n=1}^{\infty}(-1)^n J_n(k|\vec{\Delta_r}|)J_{n/2}^2(\frac{k_d |\vec{\Delta_r}|}{2})
\end{equation}

On the other hand, the spatial average of $h(\vec{r}+\vec{\Delta_r},t)h^*(\vec{r},t)$ is given by
\begin{multline}
R_r\left(\vec{\Delta r}\right) =  \lim_{D \rightarrow \infty} \frac{3}{4\pi D^3}\int_0^D\int_{-\pi}^\pi \int_{-\pi}^{\pi} \sum_{n=m=1}^N |\alpha_n|^2 e^{-j(k+k_d\cos\theta_n)|\vec{\Delta r}|\cos \psi_n}+\\ \sum_{n=1}^N\sum_{m\neq n} \alpha_n^* \alpha_m e^{j \omega_d t(\cos \theta_m - \cos \theta_n)} e^{-j(k+k_d\cos\theta_n)\left(\left|\vec{r_m} - \vec{r}\right|-\left|\vec{r_n} - \vec{r}\right|\right)}e^{-jk\left|\vec{\Delta r}\right| \cos \psi_m}r^2 \sin\theta dr d\theta d\phi
\end{multline}
Again the second term is eliminated by the integration of the complex exponential containing $\left|\vec{r_m} - \vec{r}\right|-\left|\vec{r_n} - \vec{r}\right|$.  What remains in the integrand does not depend on position, so the spatial average correlation is given by 
\begin{equation}\label{spatialcorr}
R_r\left(\vec{\Delta r}\right) =  \sum_{n=1}^N |\alpha_n|^2 e^{-j(k+k_d\cos\theta_n)|\vec{\Delta r}|\cos \psi_n}
\end{equation}

Depending on the number of scatterers $N$ and their angular distribution, Equation \ref{spatialcorr} may approach Equation \ref{finalcorr} \cite{isukapalli2006}.  However, real-world channels typically do not have sufficient scatterers to achieve this condition \cite{duel-hallen2000}.  

Spatial non-ergodicity physically means that all spatial samples observe the same channel realization.  The scatterer angles and amplitudes do not change appreciably within the collection of viewpoints.  In this case, successive spatial samples do have some mutual information, and a series of spatial samples can be used to predict the channel transfer function ahead spatially, using methods which have already been demonstrated for temporal long-range prediction \cite{duel-hallen2000}.
%this section might need some graphs in it, demonstrating how the correlation function and the spatial sample covariance are different for different situations - that might make it too long, though, and I want to focus on the other bits.
\bibliography{PL-Library}{}
\end{document}
